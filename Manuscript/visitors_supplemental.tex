
\pagebreak
%\widetext
\begin{center}
\textbf{\large Appendix B: Supplementary information on visitation}
\end{center}

\section*{Floral visitors in canola fields}

\begin{table}[h]
    \begin{tabular}{r|r|l|r|l}
     & \multicolumn{2}{c|}{Commodity fields} & \multicolumn{2}{c}{Seed fields} \\ \cline{2-5}
    Taxon & Visits & \% & Visits & \% \\ \hline
    Honey bee & 470 & 53.5 & 4850 & 77.1 \\
    Fly & 222 & 25.3 & 74 & 0.878 \\
    Hover fly & 94 & 10.7 & 151 & 1.79 \\
    Other bee & 47 & 5.35 & 30 & 0.356 \\
    Bumble bee & 25 & 2.85 & 0 & 0 \\
    Butterfly & 16 & 1.82 & 0 & 0 \\
    Leafcutter bee & 4 & 0.456 & 1675 & 19.9 \\
    % \hline
    % Total & 828 & - & 6780 & - \\
    \end{tabular}
    \caption[Flower visiting insects observed in commodity and seed canola fields]{Number of flower visitors recorded over a total of 44.8 hours of observation in commodity fields (2014 and 2015), and 46.9 hours of observation in the seed fields (2015 and 2016). ``Fly" refers to larger calyptrate muscoid flies (families Muscidae, Anthomyiidae, Caliphoridae), while ``Hover fly" refers to Syrphid flies. ``Other bee" included Halictid and Andrenid bees, while ``Bumble bee" was \textit{Bombus} spp. ``Butterfly" refers to all visiting Lepidopterans, mostly Pierids.}
    \label{tab:propVisitors}  
\end{table}

\section*{Top-working and side-working by honey bees in canola fields}

During 2015, we recorded whether honey bees were top-working or side-working flowers (see also \citealp{free1973, free1983, mohr1988}).
Top-working bees landed on the top of the flower and inserted their proboscis down between the petals to access the nectaries of the flower, while side-working bees landed on the side of the flower and stole nectar by inserted their proboscis between the petals, avoiding contact with the stigma or anthers. 
Additionally, we recorded whether honey bees were pollen or nectar foragers (pollen foragers had a visible pollen load on their corbicula, while nectar foragers had none).

\begin{table}
\begin{tabular}{r|r|l|r|l|r|l}
               & \multicolumn{2}{c|}{Commodity fields} & \multicolumn{2}{c|}{Seed fields (female bay)} & \multicolumn{2}{c}{Seed fields (male bay)} \\ \cline{2-7}
               & Top & Side & Top & Side & Top & Side           \\ \hline
Pollen forager & 44 & 2 & 12 & 0 & 115 & 0 \\
Nectar forager & 75 & 138 & 832 & 24 & 428 & 242 \\
% Total & 119 & 140 & 844 & 24 & 543 & 242
\end{tabular}
\caption[Foraging behaviours of honey bees on commodity and seed canola flowers]{Foraging behaviours of honey bees on commodity and seed canola flowers, recorded during 2015. ``Top" (top-working) indicates that the bee inserted their proboscis down between the petals from the top of the flower, while ``side" (side-working) indicates that the bee fed from the side of the flower and did not contact the anthers or stigma. Pollen foragers had pollen visible on their corbicula, while nectar foragers had none.}
\label{tab:sideWorking}
\end{table}


Pollen- and nectar-foraging honey bees had different patterns of side-working, both on commodity canola, and the male and female lines of seed canola.
Side-working was common in nectar foragers, but was more common in commodity canola (64\%) than in the male (36\%) or female bays (2.8\%) of seed canola, indicating that a large proportion of honey bees foraging on canola flowers may never come in contact with the stigmas.
Pollen foragers were almost uniformly top-foragers in both commodity and seed fields (Table \ref{tab:sideWorking}), and pollen foragers were much less common in the female bays (1.4\%) than in the male bays (15\%), or in commodity fields (18\%).
Therefore, foraging honey bees in seed canola fields tend to treat male-fertile flowers similar to commodity canola flowers, but seem to top-work flowers more in commodity canola than seed fields.
Leafcutter bee foraging behaviours were not recorded, but seemed to almost exclusively top-work flowers in seed canola fields.
